\documentclass[a4paper,12pt]{article}
\usepackage{a4wide}
\usepackage{tikz}
\usetikzlibrary{calc}

\usepackage[ngerman]{babel}

\begin{document}
\pagestyle{empty}
\setlength{\parindent}{0em}
\section*{ALU}

Ihre Aufgabe ist es, das Verhalten einer Entity  namens "`ALU"' zu programmieren. Die Entity ist in der angeh\"angten Datei "`ALU.vhdl"' deklariert und hat folgende Eigenschaften:

\begin{itemize}
\item Eing\"ange: Clk und enable vom Typ std\_logic; dies sind Takt- und Enable-Signal
\item Eing\"ange: A und B vom Typ std\_logic\_vector; dies sind Operanden
\item Eingang: slc vom Typ std\_logic\_vector; dieses Signal dient dazu, einen der verf\"ugbaren Operatoren auszuw\"ahlen
\item Ausgang: R vom Typ std\_logic\_vector; dies ist das Ergebnis der Operation mit der selben L\"ange wie die Eing\"ange
\item Ausgang: flag vom Typ std\_logic; dies ist ein Ein-Bit Flag
\end{itemize}
\vspace{0.3cm}
\begin{center}
\begin{tikzpicture}
\draw node [draw,rectangle, minimum height=35mm, minimum width=35mm,rounded corners=2mm,thick](entity){};
\draw[->,thick] ($ (entity.west)-(10mm,-12mm)$) -- ($ (entity.west) - (0mm,-12mm)$);
\draw node at ($ (entity.west)-(12mm,-12mm)$){A};
\draw[->,thick] ($ (entity.west)-(10mm,-4mm)$) -- ($ (entity.west) - (0mm,-4mm)$);
\draw node at ($ (entity.west)-(12mm,-4mm)$){B};

\draw[->,thick] ($ (entity.west)-(10mm,7mm)$) -- ($ (entity.west) - (0mm,7mm)$);
\draw node at ($ (entity.west)-(15mm,7mm)$){Clk};
\draw[->,thick] ($ (entity.west)-(10mm,12mm)$) -- ($ (entity.west) - (0mm,12mm)$);
\draw node at ($ (entity.west)-(17mm,12mm)$){enable};

\draw[->,thick] ($ (entity.east) + (0mm,-3mm)$) -- ($ (entity.east) + (10mm,-3mm)$);
\draw node at ($ (entity.east) + (14mm,-3mm)$){flag};
\draw[->,thick] ($ (entity.east) + (0mm,3mm)$) -- ($ (entity.east) + (10mm,3mm)$);
\draw node at ($ (entity.east) + (12mm,3mm)$){R};

\draw[->,thick] ($ (entity.north) + (0mm,7mm)$) -- (entity.north) ;
\draw node at ($ (entity) + (0mm,27mm)$){slc};


\draw node at ($ (entity) - (0,22mm)$){ALU};

\end{tikzpicture}
\end{center}

Ver\"andern sie die Datei "`ALU.vhdl"' nicht!
\\

Die Entity "`ALU"' soll folgendes Verhalten aufweisen:\\
Sie hat zwei Eingangsdaten (welche wir vorzeichenlos angennehmen), ein Taktsignal, ein Enable-Signal und einen Selektor, mit welchem der durchzuf\"uhrende Befehl bestimmt wird.
Der Ausgang ver\"andert sich nur bei steigender Flanke des Taktsignals. Die Datenl\"ange betr\"agt 4 Bit. Falls das Ergebnis der Operation 5 Bit gro"s sein sollte, so werden nur die niedrigsten 4 Bit an den Ausgang (Port R) weitergegeben. Das Befehls-Set der "`ALU"' beinhaltet {{INS1}}, {{INS2}}, {{INS3}} und {{INS4}}. Der Wert des Selektors f\"ur den jeweiligen Befehl lautet wie folgt:
\\

\begin{itemize}
\item "00": {{INS1}}
\item "01": {{INS2}}
\item "10": {{INS3}}
\item "11": {{INS4}}
\end{itemize}
\vspace{0.3cm}

Die Befehle sind wie folgt definiert:
\begin{itemize}
\item {{INS1}}: {{DESC1}}
\item {{INS2}}: {{DESC2}}
\item {{INS3}}: {{DESC3}}
\item {{INS4}}: {{DESC4}}
\end{itemize}
\vspace{0.3cm}

Wenn die ALU deaktiviert ist, dann sollen die Ausg\"ange (R und flag) auf 0 liegen. Ist die ALU aktiviert, so soll der entsprechende Befehl (basierend auf slc) ausgef\"uhrt und am Ausgang das Resultat ausgegeben werden. Die L\"ange des Resultats ist gleich wie die L\"ange der Eing\"ange. Au"serdem gibt es einen weiteren Ausgang, welcher ein Ein-Bit Flag darstellt und f\"ur jede ausgew\"ahlte Operation individuell berechnet werden soll. Der Wert des Flags ist von dem Ergebnis der jeweiligen Operation abh\"angig und ist wie folgt definiert:
\\
\begin{itemize}
\item {{INS1}} --\textgreater \enspace {{FLAG1}}: {{DESCflag1}}
\item {{INS2}} --\textgreater \enspace {{FLAG2}}: {{DESCflag2}}
\item {{INS3}} --\textgreater \enspace {{FLAG3}}: {{DESCflag3}}
\item {{INS4}} --\textgreater \enspace Flag: {{DESCflag4}}
\end{itemize}
\vspace{0.3cm}

Programmieren Sie dieses Verhalten in der angeh\"angten Datei "`ALU\_beh.vhdl"'.\\

Um Ihre L\"osung abzugeben, senden Sie ein E-Mail mit dem Betreff "`Result Task {{TASKNR}}"' und Ihrer Datei "`ALU\_beh.vhdl"'  an {{SUBMISSIONEMAIL}}.

\vspace{0.7cm}

Viel Erfolg und m\"oge die Macht mit Ihnen sein.

\end{document}\grid\grid
\grid
\grid
\grid
\grid
