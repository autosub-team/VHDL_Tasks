\documentclass[a4paper,12pt]{article}
\usepackage{a4wide}
\usepackage{tikz}
\usetikzlibrary{calc}
\usepackage{hyperref}

\usepackage[ngerman]{babel}

\begin{document}
\pagestyle{empty}
\setlength{\parindent}{0em}
\section*{ROM (Instruktionsspeicher)}


Ihre Aufgabe ist es, das Verhalten einer Entity  namens "`ROM"' zu programmieren. Die Entity ist in der angeh\"angten Datei "`ROM.vhdl"' deklariert und hat folgende Eigenschaften:
\begin{itemize}
\item Eing\"ange: Clk und enable vom Typ std\_logic; dies sind das Takt- und das Enable-Signal
\item Eingang: addr vom Typ std\_logic\_vector; dies ist die Adresse jener Instruktion, die ausgelesen werden soll
\item Ausgang: output vom Typ std\_logic\_vector; dies ist die ausgelesene Instruktion
\end{itemize}
\vspace{0.3cm}
\begin{center}
\begin{tikzpicture}
\draw node [draw,rectangle, minimum height=25mm, minimum width=35mm,rounded corners=2mm,thick](entity){};
\draw[->,thick] ($ (entity.west)-(10mm,-7.5mm)$) -- ($ (entity.west) - (0mm,-7.5mm)$);
\draw node at ($ (entity.west)-(15mm,-7.5mm)$){addr};
\draw[->,thick] ($ (entity.west)-(10mm,2mm)$) -- ($ (entity.west) - (0mm,2mm)$);
\draw node at ($ (entity.west)-(15mm,2mm)$){Clk};
\draw[->,thick] ($ (entity.west)-(10mm,8mm)$) -- ($ (entity.west) - (0mm,8mm)$);
\draw node at ($ (entity.west)-(17mm,8mm)$){enable};

\draw[->,thick] ($ (entity.east) + (0mm,0mm)$) -- ($ (entity.east) + (10mm,0mm)$);
\draw node at ($ (entity.east) + (17mm,0mm)$){output};



\draw node at ($ (entity) - (0,15mm)$){ROM};

\end{tikzpicture}
\end{center}

Ver\"andern sie die Datei "`ROM.vhdl"' nicht!\\

Die Entity besitzt drei Eing\"ange: Die Adresse addr, den Takt Clk und das Enable-Signal enable. Die L\"ange der Adresse betr\"agt {{ADDRLENGTH}} Bit. Die L\"ange der Instruktion betr\"agt {{INSTRUCTIONLENGTH}} Bit und ist gleich mit der L\"ange des Ausgangs output.\\

Sie sollen den VHDL Code des ROM Speichers programmieren und ihn mit der unten ersichtlichen Beispielliste an Instruktionen f\"ullen. Schreiben Sie die Instruktionen (unten in Bin\"arcode dargestellt) in der angegebenen Reihenfolge an die Speicheradressen zwischen der Startadresse {{START}} und der Endadresse {{STOP}}.\\

Instruktionen:
\begin{center}
{{DATA}}\newline
\end{center}

Beachten Sie dabei folgendes:

\begin{itemize}
\item Der Inhalt aller anderen Speicherstellen im ROM ist Null.
\item Die Entity soll als einstufiger ROM ("`single cylce ROM"') implementiert werden.
\item Der ROM arbeitet bei {{CLK}} Flanke des Taktes.
\item Wenn der ROM deaktiviert ist, soll der Ausgang auf Null gesetzt sein. Wenn der ROM aktiviert ist, dann soll die Instruktion, welche an der Eingangsadresse gespeichert ist, am Ausgang ausgegeben werden.
\end{itemize}

Programmieren Sie dieses Verhalten in der angeh\"angten Datei "`ROM\_beh.vhdl"'.\\

Um Ihre L\"osung abzugeben, senden Sie ein E-Mail mit dem Betreff "`Result Task {{ TASKNR }}"' und Ihrer Datei "`ROM\_beh.vhdl"'  an {{ SUBMISSIONEMAIL }}.

\vspace{0.7cm}

Viel Erfolg und m\"oge die Macht mit Ihnen sein.
\end{document}
